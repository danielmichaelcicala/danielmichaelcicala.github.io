\documentclass{assets/syllabus}
\begin{document}

% Output the header for the course syllabus document:
\Heading

% Do not introduce any blank lines betweeen these brackets.
\Coursehead[
% The following defines the data fields used in the document:
%
% INSTRUCTOR INFORMATION
instructor      = {Daniel Cicala},
phone           = {},
email           = {dcicala@newhaven.edu},
webpages        = {https://danielmichaelcicala.github.io/teaching},
%
% COURSE TEXT AND SYLLABUS
course          = {MATH 1117},
coursesection   = {-\,05}, % We use the - style to denote course sections.
coursetext      = {\textit{Calculus: Early Transcendentals}, by Briggs, Cochran, Gillett,  Pearson, 3e, ISBN 978-0-13-476364-45},
coursecrosslist = {},
coursename      = {Calculus I},
credits         = {4},
%
% COURSE TIME AND LOCATION
revision             = {},
semester             = {Spring 2020},
office               = {Maxcy 323},
meetingtimes         = {MWF 10:50--12:05},
classroom            = {KAPL 203},
officehours          = {MW 9--10 // WF 1--2 // F 1--3},
addanddrop           = {Wednesday 29 Jan 2020},
%semesterclasses     = {Spring Semester 2016},
startsemesterclasses = {Wednesday 22 Jan 2020},
endsemesterclasses   = {Wednesday 13 May 2020},
lastdropdate         = {Tuesday 24 Mar 2020},
%semesterholidays    = {Memorial Day Monday, May 29},
semesterfinalexam    = {Tuesday 12 May 2020}
]

\syllabusfontn
\Title{$\S$\,II: Instructors Addendum}
\SubTitle{For \Course\Coursesection \Coursename}
\vspace*{0.8em}

% ===========================================
% DISCLAIMER THAT I CAN AMEND SYLLABUS
% ===========================================

\Subtopic{Disclaimer}

I reserve the right to make changes to this
syllabus. All changes will be announced in a
timely manner through email or classroom
announcements.

% ~~~~~~~~~~~~~~~~~~~~~~~~~~~~~~~~~~~~~~~~~~~
\Subtopic{Department Syllabus}

This is the instructor portion of the syllabus.  See the
department syllabus, located in

\begin{center}
  \texttt{Blackboard > Syllabus}
\end{center}

\noindent for department policies and resources.  

% ~~~~~~~~~~~~~~~~~~~~~~~~~~~~~~~~~~~~~~~~~~~
\Subtopic{Textbook}

\Coursetext. \\

\noindent This text is not required because it will not be explicitly
referenced, though its use is encouraged to supplement learning.

% ~~~~~~~~~~~~~~~~~~~~~~~~~~~~~~~~~~~~~~~~~~~
\Subtopic{Worksheets}

During each lecture, students will complete a worksheet to
be handed in at the end of class.  It will be graded for
completeness, not correctness.  Worksheets are worth 10\% of
your total grade.

% ~~~~~~~~~~~~~~~~~~~~~~~~~~~~~~~~~~~~~~~~~~~
\Subtopic{Exercises}

Each lesson has an associated set of exercises which
students will have one week to complete two attempts. The
exercises can be found and completed in the

\begin{center}
  \texttt{Blackboard > Course Contents}
\end{center}

\noindent Exercises contribute 15\% of your total grade.


% ~~~~~~~~~~~~~~~~~~~~~~~~~~~~~~~~~~~~~~~~~~~
\pagebreak
\Subtopic{Homework}

The instructor will select one or two questions from each
section. Students will write detailed answers to
these questions which are to be graded on both the
correctness \emph{and on the quality of writing}.  These are
expected to be submitted as a ``final draft''. Handwriting
is acceptable.  While initial drafts need not be submitted,
students are encouraged to seek feedback on early drafts
during office hours.  An example homework assignment is
provided in

\begin{center}
  \texttt{Blackboard > Syllabus}
\end{center}

\noindent Homework contributes 15\% of your total grade.

% ~~~~~~~~~~~~~~~~~~~~~~~~~~~~~~~~~~~~~~~~~~~
\Subtopic{Exams}

There are two exams during the semester and one final
exam. All exams are closed notes and closed
books. Calculators and other electronics are also not
allowed. Any attempt to cheat will be dealt with according
to the misconduct code.  Students found to have violated the
rules governing academic integrity will receive a 0 grade on
the associated exam and may be subject to further penalties
as allowed by the University.  Each midterm exam contributes
15\% your total grade.  The final exam contributes 30\% of
the total grade.


% ~~~~~~~~~~~~~~~~~~~~~~~~~~~~~~~~~~~~~~~~~~~
\Subtopic{Scoring and Course Grades}

The letter grade is based on the student's total
point score for the semester.  The class letter
grade is assigned based on

% table assigning point grade to letter grade
\begin{center}
	
\syllabusfonts
\renewcommand{\arraystretch}{1.20}
\renewcommand{\colorfirst}{paleblue1}
\rowcolors{2}{\colorfirst}{white}
\begin{center}
  \begin{tabular}{|l|l|}
    \hline
    \textbf{Points} & \textbf{Grade} \\
    \hline
    97.5--100.0 & \hspace*{0.8em} \ec A\,+  \\
    92.5--97.5  & \hspace*{0.8em} \ec A     \\
    90.0--92.5  & \hspace*{0.8em} \ec A\;-- \\
    87.5--90.0  & \hspace*{0.8em} \ec B\,+  \\
    82.5--87.5  & \hspace*{0.8em} \ec B     \\
    80.0--82.5  & \hspace*{0.8em} \ec B\;-- \\
    \hline
  \end{tabular} \hspace*{1.8em}
  \rowcolors{2}{\colorfirst}{white}
  \begin{tabular}{|l|l|}
    \hline
    \textbf{Points} & \textbf{Grade} \\
    \hline
    77.5--80.0 & \hspace*{0.8em} \ec C\,+   \\
    72.5--77.5 & \hspace*{0.8em} \ec C      \\
    70.0--72.5 & \hspace*{0.8em} \ec C\;--  \\
    67.5--70.0 & \hspace*{0.8em} \ec D\,+   \\
    62.5--67.5 & \hspace*{0.8em} \ec D      \\
    60.0--62.5 & \hspace*{0.8em} \ec D-     \\
    0--60      & \hspace*{0.8em} \ec F      \\
    \hline
  \end{tabular}
  
\end{center}
\syllabusfontn
	
\end{center}

% ~~~~~~~~~~~~~~~~~~~~~~~~~~~~~~~~~~~~~~~~~~~
\Subtopic{Grade Calculation}

The total grade will be calculated using the formula
\[
  0.10 (\textrm{worksheets}) +
  0.15 (\textrm{exercises}) +
  0.15 (\textrm{homework}) +
  0.30 (\textrm{midterm exams}) +
  0.30 (\textrm{final exam})
\]

% ~~~~~~~~~~~~~~~~~~~~~~~~~~~~~~~~~~~~~~~~~~~
\pagebreak
\Subtopic{Course Outline Schedule}

\Semester\ classes are from \Startsemesterclasses\
to \Endsemesterclasses.  The last day to drop
classes without any financial penalty is
\Addanddrop, and the last day to withdraw from the
class, i.e., to request a {\ec W} grade is prior
to \Lastdropdate.  All requests for a W must be
sent to the registrar.

The following is a tentative schedule. The exam dates are
also noted on the schedule. These dates are subject to
change; this includes the exams.

\begin{center}
  \rowcolors{2}{cyan!15}{white}
  \begin{tabular}{|llll|}
    \hline
    \rowcolor{cyan!30}{\textbf{Dates}}
      & \textbf{Monday}
      & \textbf{Wednesday}    
      & \textbf{Friday} \\ 
    \hline
    {\bf Week 1}
      & no class
      & class intro
      & functions \\
    {\bf Week 2}
      & inverse functions
      & algebra of functions
      & limits graphically \\
    {\bf Week 3}
      & limits analytically
      & more limits  
      & continuity \\
    {\bf Week 4}
      & chapter 2 wrapup
      & derivative intro  
      & derivative as function \\
    {\bf Week 5}
      & derivative rules
      & more derivative rules
      & trig derivatives \\      
    {\bf Week 6}
      & derivative applications
      & \emph{exam review} 
      & \textbf{exam 1} \\
    {\bf Week 7}
      & chain rule
      & implicit differentiation 
      & deriving log, exp, arctrig \\
    {\bf Week 8}
      & related rates
      & chapter 3 wrapup
      & extrema \\
    {\bf Week 9}
      & no class
      & no class
      & no class \\
    {\bf Week 10}
      & mean value theorem
      & graphing functions 1
      & graphing functions 2\\
    {\bf Week 11}
      & optimization
      & linear approx.
      & l'hopital  \\
    {\bf Week 12}
      & antiderivatives
      & \emph{exam review}
      & \textbf{exam 2} \\
    {\bf Week 13}
      & area under curves
      & definite integrals
      & FTC \\
    {\bf Week 14}
      & working w integrals 
      & substitution
      & instructor topics \\
    {\bf Week 15}
      & instructor topics
      & instructor topics
      & instructor topics \\
    {\bf Week 16}
      & final review
      & reading day
      & no class \\
    \hline
    {\bf Final Exam} & \Semesterfinalexam \\
    \hline
  \end{tabular}
\end{center}


%COPIES OF THE SCHEDULE FOR MW AND TR ARE KEPT IN ./lib.........................
{
\renewcommand{\syllabusfontstable}{\syllabusfontf}
\renewcommand{\tableheader}{\tableheaders}
%\input{schedule.tex}
}

\LastPage

\end{document}

