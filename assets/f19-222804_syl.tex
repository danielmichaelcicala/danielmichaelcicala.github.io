\documentclass{assets/syllabus}
\begin{document}

% Output the header for the course syllabus document:
\Heading

% Do not introduce any blank lines betweeen these brackets.
\Coursehead[
% The following defines the data fields used in the document:
%
% INSTRUCTOR INFORMATION:
instructor      = {Daniel Cicala},
phone           = {},
email           = {dcicala@newhaven.edu},
webpages        = {https://danielmichaelcicala.github.io/teaching},
%
% COURSE TEXT AND SYLLABUS:
course          = {MATH 2228},
coursesection   = {-\,04}, % We use the - style to denote course sections.
coursetext      = {\emph{Introduction to the Practice of
                   Statistics}, by David Moore, 8e, ISBN
                   9781464158971 (2014)},
coursecrosslist = {},
coursename      = {Elementary Statistics},
credits         = {4},
%
% COURSE TIME AND LOCATION:
revision        = {},
semester        = {Fall 2019},
office          = {Maxcy 323},
meetingtimes    = {MWF 925-1040a},
classroom       = {KAPL 106},
officehours     = {M 2-4 // W 12-2 // F 11-1},
addanddrop           = {Tuesday, 3 September},
%semesterclasses      = {Spring Semester 2016},
startsemesterclasses = {Monday, 26 August},
endsemesterclasses   = {Wednesday, 18 December},
lastdropdate         = {Tuesday, 29 October},
%semesterholidays     = {Memorial Day Monday, May 29},
semesterfinalexam    = {Friday, 13 December, 8-10a}
]

\syllabusfontn
\Title{$\S$\,II: Instructors Addendum}
\SubTitle{For \Course\Coursesection \Coursename}
\vspace*{0.8em}

% ===========================================
% DISCLAIMER THAT I CAN AMEND SYLLABUS
% ===========================================

\Subtopic{Disclaimer}

I reserve the right to make changes to this syllabus. All
changes will be announced in a timely manner through email
or classroom announcements.

% ~~~~~~~~~~~~~~~~~~~~~~~~~~~~~~~~~~~~~~~~~~~
\Subtopic{Department Syllabus}
This is the instructor portion of the syllabus.  see the
department syllabus, located at
\begin{center} \href{https://danielmichaelcicala.github.io/assets/f19-2228_deptsyl.pdf}{https://danielmichaelcicala.github.io/assets/f19-2228_deptsyl.pdf}
\end{center}
for department policies and resources.

% ~~~~~~~~~~~~~~~~~~~~~~~~~~~~~~~~~~~~~~~~~~~
\Subtopic{Required Textbook}

\Coursetext.

% ~~~~~~~~~~~~~~~~~~~~~~~~~~~~~~~~~~~~~~~~~~~
\Subtopic{Course Website}

I will only use Blackboard to email the class and to make
announcements.  Be sure that you have emails from Blackboard
whitelisted. All other course information is on the course
website
\href{https://danielmichaelcicala.github.io/teaching}{listed
  above.}.  In particular, I will not post grades on
Blackboard, but I am always happy to discuss your grade in-person
during office hours.

% ~~~~~~~~~~~~~~~~~~~~~~~~~~~~~~~~~~~~~~~~~~~
\Subtopic{Statistical programming projects}

In this class, we will use the statistical programming
software RStudio. For this, you will occasionally need to
bring a laptop into class (see the schedule below). If you
do not have your own laptop, you can loan one from the
library.

To download RStudio, visit
\href{https://www.rstudio.com/products/rstudio/download/}{https://www.rstudio.com/products/rstudio/download/}
and select either the \emph{RStudio Desktop} or the
\emph{RStudio Server} option.  Then select the link for your
operating system.  See me during office hours if you are
having difficulty.

We will integrate RStudio into class by way of a number of
projects posted on the class website listed above. Each
project is associated to one section or chapter of the
text. Each project will be completed and turned in
induvidually, but after we complete each chapter, there will
be 30 minutes of class time where you may collaborate on the
projects with your classmates and ask me questions.
Completed project reports are to be emailed to me in a pdf
format only strictly before 4pm on the due date posted on
the class website. No late assignments (that includes those
I receive at 4:01pm) will be accepted without a university
approved reason. See the document on the course website
titled \emph{Report Submission Policy} for details on how to
write and submit your labs.

Every project will be given a grade of 0-10.  Project
reports are worth 15\% of your grade.



% ~~~~~~~~~~~~~~~~~~~~~~~~~~~~~~~~~~~~~~~~~~~
\Subtopic{Class Participation}

During class, please keep a separate clean sheet of paper
accessable.  Throughout the lectures, I will pose questions,
for example, asking you to guess the answer to something I
haven't taught yet. I will provide a short time for you,
ideally in collaboration with your classmates, to consider
the question. In this time, write down your thoughts on that
paper.  You will hand in this paper at the end of class; it
counts as your attendance. It is otherwise not graded as it
is meant for you to play with your ideas, no matter how
stupid *you* think they are.

The reason we'll do this is because the ability to make
predictions about material demonstrates a deeper
understanding than simply recalling facts. But, like
anything, this requires practice.

Class participation is worth 5\% of your total grade.

% ~~~~~~~~~~~~~~~~~~~~~~~~~~~~~~~~~~~~~~~~~~~
\Subtopic{Homework}

On my website listed above, you will find a number of
homework problems and their due date. They are to be handed
in to me or placed in the folder outside my office by 4pm on
the due date. Late assignments will not be accepted without
a university approved reason. Assignments that are not
stapled or without names will not receive credit.

Each homework will be given a score between 0 and 10 that
corresponds with the percentage of the assignment you
completed. For a question to count as ``complete'', you must
either have provided a full, though not necessarily correct,
answer. In lieu of a full answer, you can either (1) discuss
the question with me in my office hours prior to the due
date of the related assignment, or (2) write a short
paragraph that examines exactly why you had difficulty and
what you tried to do when you got stuck.

Math is a *skill* that can be developed and like all
skills, it requires practice.  It is impossible to get
better without making and learning from mistakes.  By
developing the ability to recognize your mistakes, then soon
enough you will catch them in real-time and prevent making
them all together. This homework is to allow you to be wrong
while not hurting your grade.  I will offer vague comments
to direct your attention to areas of your work where I think
you can do better. You should study these areas to better
understand them. I encourage you to come to office hours to
discuss any of my comments.

Homework contributes 10\% of your total grade.


% ~~~~~~~~~~~~~~~~~~~~~~~~~~~~~~~~~~~~~~~~~~~
\Subtopic{Quizzes}

On each day when a homework is due, there will be a short
quiz. This quiz will consist of questions pulled *directly*
from the assigned homework. You may not use your text,
notes, or any electronics apart from a scientific
calculator. Phone calculators are not allowed. Each
question, graded for correctness and clarity, will be given
a score between 0 and 10.

These quizzes are to practice your ability to recall
mathematical information. This ability---which, again, is a
skill to be developed---serves as a frame onto which new
knowledge is built.  Also, to solve a problem, recall is needed
for you to sythesize the information into a solution.  You
cannot synthesize information held in a second-brain, e.g.~a
notebook or the internet. Quizzes are a low-stakes way for
you to practice your ability to access information.

Quizzes are worth 10\% of your total grade.

% ~~~~~~~~~~~~~~~~~~~~~~~~~~~~~~~~~~~~~~~~~~~
\Subtopic{Exams}

There are two exams during the semester and one final
exam. All exams are closed notes and closed
books. Scientific calculators are permitted but all other
electronics are not allowed. Any attempt to cheat will be
dealt with according to the misconduct code.  Students found
to have violated the rules governing academic integrity will
receive a 0 grade on the associated exam and may be subject
to further penalties as allowed by the University.

Not only are exams meant to measure the breadth of knowledge
you've obtained, but they are actually an important part of
*enhancing* your knowledge. 

Each midterm exam contributes 15\% your total grade.  The
final exam contributes 30\% of the total grade. 



% ~~~~~~~~~~~~~~~~~~~~~~~~~~~~~~~~~~~~~~~~~~~
\Subtopic{Grade Calculation}

The total grade will be calculated using the formula
\[
  0.15 ( \textrm{reports} )
  0.05 (\textrm{class part.}) +
  0.10 (\textrm{hw}) +
  0.10 (\textrm{quizzes}) +
  0.30 (\textrm{midterm exams}) +
  0.30 (\textrm{final exam})
\]


% ~~~~~~~~~~~~~~~~~~~~~~~~~~~~~~~~~~~~~~~~~~~
\Subtopic{Scoring and Course Grades}

The letter grade is based on the student's total point score
for the semester.  The class letter grade is assigned based
on

% table assigning point grade to letter grade
\begin{center}
	
\syllabusfonts
\renewcommand{\arraystretch}{1.20}
\renewcommand{\colorfirst}{paleblue1}
\rowcolors{2}{\colorfirst}{white}
\begin{center}
  \begin{tabular}{|l|l|}
    \hline
    \textbf{Points} & \textbf{Grade} \\
    \hline
    97.5--100.0 & \hspace*{0.8em} \ec A\,+  \\
    92.5--97.5  & \hspace*{0.8em} \ec A     \\
    90.0--92.5  & \hspace*{0.8em} \ec A\;-- \\
    87.5--90.0  & \hspace*{0.8em} \ec B\,+  \\
    82.5--87.5  & \hspace*{0.8em} \ec B     \\
    80.0--82.5  & \hspace*{0.8em} \ec B\;-- \\
    \hline
  \end{tabular} \hspace*{1.8em}
  \rowcolors{2}{\colorfirst}{white}
  \begin{tabular}{|l|l|}
    \hline
    \textbf{Points} & \textbf{Grade} \\
    \hline
    77.5--80.0 & \hspace*{0.8em} \ec C\,+   \\
    72.5--77.5 & \hspace*{0.8em} \ec C      \\
    70.0--72.5 & \hspace*{0.8em} \ec C\;--  \\
    67.5--70.0 & \hspace*{0.8em} \ec D\,+   \\
    62.5--67.5 & \hspace*{0.8em} \ec D      \\
    60.0--62.5 & \hspace*{0.8em} \ec D-     \\
    0--60      & \hspace*{0.8em} \ec F      \\
    \hline
  \end{tabular}
  
\end{center}
\syllabusfontn
	
\end{center}

% ~~~~~~~~~~~~~~~~~~~~~~~~~~~~~~~~~~~~~~~~~~~
\Subtopic{Course Outline Schedule}

\Semester\ classes are from \Startsemesterclasses\ to
\Endsemesterclasses.  The last day to drop classes without
any financial penalty is \Addanddrop, and the last day to
withdraw from the class, i.e., to request a {\ec W} grade is
prior to \Lastdropdate.  All requests for a W must be sent
to the registrar.

The following is a tentative schedule. The exam dates are
also noted on the schedule. These dates are subject to
change; this includes the exams.

\begin{center}
  \rowcolors{2}{cyan!15}{white}
  \begin{tabular}{|l l|}
    \hline
    \rowcolor{cyan!30} \textbf{Dates} & \textbf{Section} \\ 
    \hline
    {\bf Week 1}  & 1.1, 1.2, 1.3 \\
    {\bf Week 2}  & 1.3(cont.),  \\
    {\bf Week 3}  & 2.1, 2.2, 2.3 \\
    {\bf Week 4}  & 2.3, 2.4, 2.5 \\
    {\bf Week 5}  & 2.5 (cont.), 2.6, 3.1 \\
    {\bf Week 6}  & 3.2, 3.3, 3.4 \\
    {\bf Week 7}  & 4.1, 4.2, 4.3 \\
    {\bf Week 8}  & 4.4, 4.5, exam1 \\
    {\bf Week 9}  & 5.1, 5.2, 6.1 \\
    {\bf Week 10} & 6.2, 6.3, 6.4 \\
    {\bf Week 11} & 7.1, 7.2, 7.3 \\
    {\bf Week 12} & exam2, 8.1, 8.2 \\
    {\bf Week 13} & 9.1, 9.2, 9.3, 10.1, 10.2 \\
    {\bf Week 14} & 11.1, 11.2, 12.1, 12.2, 13.1, 13.2 \\
    \hline
    {\bf Final Exam} & \Semesterfinalexam \\
    \hline
  \end{tabular}
\end{center}


%COPIES OF THE SCHEDULE FOR MW AND TR ARE KEPT IN ./lib.........................
{
\renewcommand{\syllabusfontstable}{\syllabusfontf}
\renewcommand{\tableheader}{\tableheaders}
%\input{schedule.tex}
}

\LastPage

\end{document}
